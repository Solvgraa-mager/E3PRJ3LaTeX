\section{Projektets mål} 

Projektets overordnede mål tager lige dele udspring i et ønske om et underholdende produkt, som i en grad kan leve videre efter semesterets afslutning såvel som gruppens fælles specifikke teknologiske interesser.
Produktets vision er et konkurrencepræget spil baseret på sumobrydning mellem to robotter i en glasmontre, hvor robotternes retning og hastighed styres på en hidtil ukendte måder – herunder analyse af lydniveauer og frekvenser som produceres af de to spillere såvel som sensorinput af andre arter.
Lyden produceres f.eks. med en blokfløjte eller simpelthen ved stemmens kraft.
Gruppens ønske om undersøgelse af teknologier indenfor analog- og digital lyd behandling indfries naturligt igennem robotternes styringsenheder og ønsket om dybere indsigt i design af software til indlejrede systemer, samt kommunikation mellem disse, indfries igennem hhv. robotterne- og controllernes system- og netværks-design.
Projektet vil yderligere bibringe en stor læring om aktuatorer, mere specifikt motorstyring, som vil være en central del af projektet.

Foruden de faglige mål for læring ønsker gruppen at styrke kendskabet til Scrum som vil anvendes til at styre en iterativ arbejdsproces, som adskiller sig fra processen kendt fra tidligere semestre. 

