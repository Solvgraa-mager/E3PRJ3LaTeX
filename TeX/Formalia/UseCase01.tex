%Sometimes it is a good idea to put domain objects in \texttt{}
%The template and the descriptions are based on the book Applying UML and Patterns: 
%An Introduction to Object-Oriented Analysis and Design and Iterative Development
%(3rd Edition) by Craig Larman.

%%% USE CASE 1
\subsubsection{Use Case 1}
Se \tabref{usecase:1}, hvor Use Case 1: "Styr \gls{sumobot} med joystick" beskrives. 
\begin{usecase}{Fully dressed beskrivelse af Use Case 1: "Styr \gls{sumobot} med joystick".}{1}
  \addtitle{Use Case 1}{"Styr \gls{sumobot} med joystick"}
  %Level: "user-goal" or "subfunction"
  \addfield{Mål:}{At en \gls{sumobot} bevæger sig på spillebanen baseret på joystickinput.}
  %Level: "user-goal" or "subfunction"
  \addfield{Initiering:}{Spiller bruger joysticket som styringsenhed}

  %Primary Actor: Calls on the system to deliver its services.
  \addfield{Primær Aktør:}{Bruger}
  \additemizedfield{Sekundær Aktør:}{
    \item \gls{sumobot}
    \item Styringsenhed
    \item Central computer
  }

  \addfield{Antal samtidig forekomster:}{2}

  %Preconditions: What must be true on start and worth telling the reader?
  \addfield{Prækondition:}{\gls{rsb}-spillet er initieret \tbr}
  %when multiple
  %\additemizedfield{Preconditions:}{} 

  %Postconditions: What must be true on successful completion and worth telling the reader
  \additemizedfield{Postkonditioner:}{
    \item \gls{sumobot} har bevæget sig på baggrund af styringsinput fra joysticket.
    \item Afventer nyt input.
  }
  %when multiple
  %\additemizedfield{Preconditions:}{}

  %Main Success Scenario: A typical, unconditional happy path scenario of success.
  \addscenario{Hovedscenarie:}{
    \item Brugeren tilgår joystick-styringsenhed for spiller 1
    \item Brugeren bevæger joysticket
    \item Styringsenheden behandler bevægelsen
    \item[] [Ext01: Bruger styrer hastighed]
    \item[] [Ext02: Bruger styrer retning]
    \item Styringsenhed overfører data til central computer
    \item Central computer behandler styringsdata
    \item Central computer overfører styringsdata til \gls{sumobot}
    \item \gls{sumobot} udfører bevægelse.
  }
  \addscenario{Udvidelser / Undtagelser:}{
    \item[Ext01:] Bruger styrer hastighed
    \item[1.] Bruger tilter joystick på $y$-aksen.
    \item[2.] Use case fortsætter fra punkt 3\newline
    \item[Ext02:] Bruger styrer retning:
    \item[1.] Bruger tilter joystick på $x$-aksen.
    \item[2.] Use case fortsætter fra punkt 3
  }
\end{usecase}\label{tab:UseCase:1}
