\section{SumoBot analyse}


\subsubsection{Interfacebeskrivelse}

\begin{table*}[]
        \centering
        \caption{Interfacebeskrivelse for SumoBot}
        \label{tab:interface_table_SumoBot}
        \begin{tabular}{lp{5cm}p{7cm}}\toprule 
                \textbf{Navn}          & \textbf{Type} & \textbf{Beskrivelse}                      \\
                in/out SumoBotKom      & SumoBotKom    & WiFi\tbr signal fra Central Computer.     \\
                in/out SumoBotCMD      & SumoBotCMD    & Signal til Microkontroller.               \\
                
                in pakoersel            & Force         & Fysisk påvirkning fra omverdenen.        \\
                out pakoerselSignal     & digital       & Logisk I/O signal.                       \\
                
                pakoerselRegistrerin    & Digital       & Logisk I/O signal.                       \\
                out motorCmdOutLeft    & Seriel        & Signal til motorstyring om venstre motor. \\
                out motorCmdOutRight   & Seriel        & Signal til motorstyring om højre motor.   \\
                
                in motorCommandinLeft  & Seriel        & Signal til motorstyring om venstre motor. \\
                in motorCommandInRight & Seriel        & Signal til motorstyring om højre motor.   \\
                out motorCtrOut[2]     & PWM           & PWM signal til motor.                     \\
                
                in motorCtrIn[2]       & PWM           & PWM signal til motor.                     \\
                out motorTorqueOut     & torque        & Energi fra motor til hjul.                \\
                
                in ChargePowerIn       & DC            & DC strøm fra oplader til batteri.         \\
                in powerBat            & DC            & 12V\tbr fra batteri.                      \\
                in MCPower             & DC            & 5V Forsyning.                             \\
                out PowerOut           & DC            & 5V forsyning.                             \\
                
                
                \bottomrule
        \end{tabular}%
\end{table*}


\subsubsection*{\textbf{Central Computer IF}}
Dette er kommunikationsledet mellem SumoBot og Central Computeren. Det er denne der sørger for at fortælle SumoBotten omkring retning og fart, 
Her er det tænkt at en raspberry pi zero skal anvendes da den har indbygget WiFi og Bluetooth, som er 2 gode trådløse kommunikations former der vil passe til projektet i forhold til rækkevidde og hastighed. 

Fordele: 
\begin{itemize}
\item 1. Lille og kompakt enhed med gode trådløse egenskaber.
\item 2. Kører 5V, så er kompatibel med resten af systemet.
\item 3. Har huller til montering i hver hjørne.
\end{itemize}

Ulemper: 
\begin{itemize}
\item 1. Har HDMI udgange samt USB som ikke bliver anvendt, og derfor tager unødig plads.
\end{itemize}

Konklusion:

\subsubsection*{\textbf{Attack Sensor}}
Til registrering af sammenstød mellem robotterne, skal der implementeres en form for kontaktsensor.
Her er der tænkt at anvende DM1-03P-30-3  som er en vippekontakt med en høj følsomhed\cite{DM1-03P-30-3Data}.
Fordele: 
\begin{itemize}
\item 1. Lille, kompakt og følsom kontakt.
\item 2. Kan håndtere 5V på 30mA som er nok til at lave et logisk høj på en Microkontroller.
\end{itemize}

Ulemper: 
\begin{itemize}
\item 1. Kan ikke modstå stød fra alle vinkler uden risiko for beskadigelse.
\end{itemize}

Konklusion:
Det er ikke tiltænkt at et SumobotBattle-spil skal varer i længere tid. Derfor vægtes følsomheden af sensoren højt, da en følsom attack sensor vil resultere i hurtigere runder i og med at angreb indtræffer oftere. Der vurderes at "DM1-03P-30-3" er egnet til opgaven. Den mindre robusthed tænkes at kompensere for ved at vende sensoren lodret således at risikoen for ikke-optimale berøringer begrænses.
% Hvad med andre kontaktformer?

\subsubsection*{\textbf{PSU}}
Det er SumoBottens PSU der skal levere et stabilt 5V DC signal til de interne komponenter. Det er vigtigt at den kan levere et stabilt signal så komponenterne ikke slukker eller genstarter pga. for høj eller lav spænding.  

Fordele: 
\begin{itemize}
\item 1. Laver et stabilt 5V DC signal "+- 0.25V" !!DATABLAD HENVISNING!! som komponenterne i SumoBotten anvender.
\item 2. Fylder meget lidt. 
\end{itemize}

Ulemper: 
\begin{itemize}
\item 1. Den er kun fuld funktionel og stabil inden for 7 V til 20 V DC.
\end{itemize}

Konklusion:


\subsubsection*{\textbf{Microcontroller}}
Mikrokontrolleren er den del af SumoBotten der styrer motorstyringen ud fra indput fra Central Computer IF. Det er også denne der holder øje med Attack Sensoren og sender dataen videre til Central Computer IF. Derfor skal den være hurtig nok til at bearbejde dataen og modtage/sende det videre samt styre en motorstyring. Her er det også vigtigt at den er kompatibel med en RPI, da det er denne der bliver anvendt som Central Computer IF. 

Udfra disse oplysninger er der valgt at anvende en "CY8CKIT-059 PSoC 5LP". 

Fordele: 
\begin{itemize}
\item 1. Lille og kompakt.
\item 2. Har tilstrækkeligt med ben til kommunikation med motorstyring, attack sensor og central computer IF.
\item 3. Anvender 5V til forsyning og er meget strømeffektiv.
\end{itemize}

Ulemper: 
\begin{itemize}
\item 1. Har ikke monterings huller, eller andre former for monteringsmuligheder udover ben.
\item 2. Skal have forloddet ben monteret for at virke.
\item 3. Der skal være frit til dens USB udgang for programmering af microcontrolleren. 
\end{itemize}

Konklusion:


\subsubsection*{\textbf{Batteri}}
I og med at Sumobot er trådløs, er det essentielt at have et batteri monteret. Men da robotten har en fysisk begrænsning, må batteriet ikke være for stort. Derudover skal batteriet levere et hvis spændingsniveau, som kan forsyne de forskellige delmoduler.
Der er tideligere bestemt at der vil blive anvendt en spændingsregulator LM7805, som kræver en input spænding på 7-20 V. En paramter der også skal inddrages i overvejelserne til valg af batteri.

Fordele: 
\begin{itemize}
\item 1.
\item 2.
\end{itemize}

Ulemper: 
\begin{itemize}
\item 1.
\item 2.
\end{itemize}


\subsubsection*{\textbf{Motor}}
I valg af motor sætter de fysiske dementioner (20x20x20) af SumoBot atter en begrænsning. Motorene må derfor ikke overgå en længde på over 8 cm, så der er plads til to 2 stk. motor. Et andet aspekt der er værd at tage i betragtning er momentet i valget af motorer. På nuværende tidspunkt er den samlede vægt endnu ukendt, dog kan man give et skøn vurderet efter de maksimale fysiske mål for Sumobot. Valg af motor bygger derfor også på om motoren er i stand til at flytte robotten. \tbd


Fordele:\tbr 
\begin{itemize}
\item 1.
\item 2.
\end{itemize}

Ulemper: \tbr
\begin{itemize}
\item 1.
\item 2.
\end{itemize}

Konklusion:
\tbr
\subsubsection*{\textbf{Motorstyring}}
Motorstyringen tager sig af styring af de 2 fastmonterede motore på SumoBotten. Der skal være mulighed for at køre de 2 motore begge retninger med forskellige hastigheder uafhængige af hinanden. Den skal også kunne holde til det "load" motorene påtrykker.

Her er der valgt at anvende en L298N, som er en dual full bridge driver\cite{L298Data}. Dette betyder at begge motorene kan styres uafhængigt af hinanden ved at avende 1 logisk kreds.\tbr

Fordele: 
\begin{itemize}
\item 1. Kan anvendes til styring af begge motorer.
\item 2. Kan køre med 5V logistik.
\item 3. Kan køre fra 4.8 V til 46 V DC med 2A på indgang / udgang til motoren.
\item 4. Turn off / on delay og rise / fall time på under 2.1 mikro sekunder.
\item 5.  Meget kompakt enhed. 
\end{itemize}

Ulemper: 
\begin{itemize}
\item 1. Ingen
\end{itemize}

Konklusion:


\subsubsection{Analysetanker}

\textbf{Motor:}
- Da de fysiske mål (max) for vores sumoBot er 20x20x20 cm, skal vores motor være i stand til at kunne trække netop denne vægtbelastning.
- Da motorene skal sidde vandret på langs ved siden af hinanden må motor + dæk ikke være længere end 8 cm pr. motorenhed.
- Når vi skal vælge motor, skal vi også have i mente, at hver bil består af 2 motorer. Vi skal altså i alt bruge 4 stk. Det betyder at valget af motor også afhænger af tilgængeligheden af netop den motormodel.

\textbf{Batteri:}
- skal køre mellem 7 til 12 V.
- Vores batteri skal forsyne andet end bare motoren. Der skal altså være en form for spændingsregulering til de enkelte delelementer.
- Under PRJ1 anvendte vi et batteri på 9,6V, som efterfølgende blev reguleret ned til 5v med en LM7805\cite{LM78xxData}. Vores erfaring med den spændingsregulator var at den gav et stabilt spændingsoutput når inputtet var 7+ volt.
- Da motor og styring kører på samme batteri skal fly-back dioder. stabilisator kondensatorer anvendes. 

\textbf{Dæk:}
- Dækkene må ikke have en større radius end 5 cm.

\textbf{Motorstyring "pr. bil":}
- 2x H-bro til styring af motorretning.
- 2x Mosfet til PWM styring af hastighed.
- Flyback dioder til beskyttelse af batteri.


\textbf{Microcontroller:} - SPØRGSMÅL TIL CENTRAL PC HOLD
- PSoC til styring af Hbro og attack sensor.
- RPI til trådløs kommunikation.

Attack sensor:
- Knap med maks diameter på 1 cm.
- Skal kunne registrere at den anden bil rammer den. "følsomhed"

\textbf{Bilkarosseri:}
- Alle dele skal påmonteres på et karosseri, altså en platform der har plads til både batteri, motorer, attacksensor, MCU, PSU og dæk.
Under 1. semesterprojekt, hvor vi også skulle bygge en bil, fik vi givet et chassis på forhånd med påmonteret hjul, aksler, motor osv. På den måde sparede vi tid, da det mekaniske allerede var udviklet, og vi kunne derfor fokusere på vores eget fagområde, det elektroniske aspekt.
Denne gang står vi uden færdigbygget karosse, og vi ser to muligheder:
\begin{enumerate}
    \item Vi tegner en model i et 3D-værktøj og får det CNC-fræset af Rasmus Elm fra Protolab\footnote{https://ase.au.dk/om-ingenioerhoejskolen/laboratorier-og-vaerksteder/katrinebjerg/protolab-shannon/}. Da ingen i gruppen har erfaring i at tegne i 3D-programmer, kan der være en stejl indlæringskurve, altså at processen er langsommelig, og vi ikke ønsker at bruge for meget tid på noget, der egentlig ikke er det vi bliver bedømt på. % Jeg har erfaring, dog ikke i Solidworks - Adam
\item Søgen i genbruger og eller andre butikker efter en allerede eksisterende bil, der overholder de fysiske kriterier for dimensioner. Der kan herefter afmonteres diverse dele, hvorved et bart karosseri med motor, aksler og hjul er klar til berarbejdelse.
\end{enumerate}
Sidstnævnte mulighed kan spare os for meget arbejde. Der er dog en hvis risiko hvis vi "satser" på dette valg, i og med at der er en sandsynlighed for at vi ikke kan finde noget køretøj, der tilfredsstiller vores ønsker og overholder de krav vi har fået stillet.
