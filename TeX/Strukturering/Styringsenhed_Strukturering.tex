\section{Styringsenhed}\label{sec:Styringsenhed:Strukturering}

For Styringsenheden er der udviklet struktur som illustreres ved \figref{Strukturering/Styringsenhed/XXXXBDDXXXX}, og \figref{Strukturering/Styringsenhed/IBD_styringsenhed_rev2}. 
Delblokkene og interfaces imellem disse, som skitseret i IBDet \figref{Strukturering/Styringsenhed/IBD_styringsenhed_rev2} er yderligere beskrevet i blokbeskrivelsen som findes i 
\tabref{blok_table_Robo-Sumo-Battle_blokbeskrivelse}.

\fig{Strukturering/Styringsenhed/IBD_styringsenhed_rev2}{1}{IBD for Styringsenhed}

\begin{BlokDescription}{Styringsenhed [1/2]}{Styringsenhed_1}
%\Blokbeskrivelse{Bloknavn}{Funktion}
%\Portbeskrivelse{Portnavn}{Type}{Beskrivelse}
\Blokbeskrivelse{Analog signalbehandling}{Forstærker mikrofon inputtet i en grad at ADC kan konvertere de forskellige værdier. \newline Gain: 250gg}
\Portbeskrivelse{Lydsignal}{Analog}{Ufiltreret og forstærket signal indeholdende lydinformation.}{
\item Indgangsimpedans: Høj
}
\Portbeskrivelse{Behandlet lydsignal}{Analog}{Filtreret og forstærket signal indeholdende lydinformation.}{
\item Signalfrekvensområde: \SIrange{300}{3e3}{\Hz}
\item Udgangsimpedans: Lav
}
\BlokSpacer{0cm}
%%%%%%%%%%%%
\Blokbeskrivelse{Mikrofon}{Konvertering fra akustisk til elektrisk signal}
\Portbeskrivelse{Mikrofon}{Analog}{Elektret mikrofon.}{
\item Udgangsimpedans: 2.2\kohm 
\item Sensitivitet: \SI{8(2)}{\milli\volt\per\pascal}
\item SNR: 40dBV
}
\Portbeskrivelse{SpillersLyd}{Akustisk}{Fra mundharmonika.}{
\item Dynamisk område: 70dB SPL
\item Signalfrekvensområde: \SIrange{250}{1550}{\Hz}
}
\Blokbeskrivelse{Signal\newline konditionering}{Buffer stage der skal sikre maksimal spændingsoverførsel.}
\Portbeskrivelse{Behandlet lydsignal}{Analog}{}{
\item Indgangsimpedans: 10x udgangsimpedans for Analog Signalbehandling blokken.
}

\Blokbeskrivelse{ADC}{Analog til digital konvertering.}
\Portbeskrivelse{ADCinLyd}{Analog}{}{
\item Fuld skala input spændingsområde: \SIrange{0}{5}{\volt}
\item Sample rate: 10 kHz.
\item Bitopløsning: 8 bit
\item SNR: 40 dBV
\item Indgangsimpedans: Høj
}

\Blokbeskrivelse{Digital Signal Processing}{Processering af samples fra ADC}
\Portbeskrivelse{DSPin}{Digital}{Array med outputsamples fra ADC}{
\item Type: uint8 (8 bit samples)
\item Antal samples: 512
}
\Portbeskrivelse{Styringsprotokol}{SPI}{Retning og hastighed for SumoBot}{
\item Se \nameref{protokolbeskrivelse}
}
\end{BlokDescription}

\subsubsection{Protokolbeskrivelse for styringsprotokol}\label{protokolbeskrivelse}
I dette afsnit klarlægges protokollen for styrings af SumoBotsne. Protokollen i \tabref{protokol:d7d0} hele byten hvorved styringskommandoen ligger i. I \tabref{protokol:truthtable} ses en sandhedstabel for hvorledes encodningen er tiltænkt for databits D6-D4 (hastighed) samt D2-D0 (retning).
Disse tabeller er understøttet af \figref{Strukturering/Styringsenhed/StyringsenhedCirkelUdkast} som illustrerer hvordan en databyte fortolkes af en SumoBot. 

\begin{table}[!h]
\centering\small 
\caption{Byte med styringsprotokol}
\label{tab:protokol:d7d0}
\begin{tabular}{@{}llllllll@{}}
D7 & D6 & D5 & D4 & D3 & D2 & D1 & D0 \\\midrule
\makecell[ll]{Frem = 1 \\Tilbage = 0} & X  & X  & X  & \makecell[ll]{Højre = 1 \\Venstre = 0}& X  & X  & x \\\bottomrule
\end{tabular}%
\end{table}

\begin{table}[!h]
\centering\small 
\caption{Sandhedstabel for hastighed [D6-D4] og retning [D2-D0]}
\label{tab:protokol:truthtable}
\begin{tabular}{@{}llll@{}}\toprule
D6 & D5 & D4 & PWM    \\\midrule
1  & 1  & 1  & 100\%  \\
1  & 1  & 0  & 87,5\% \\
1  & 0  & 1  & 75\%   \\
1  & 0  & 0  & 62,5\% \\
0  & 1  & 1  & 50\%   \\
0  & 1  & 0  & 37,5\% \\
0  & 0  & 1  & 25\%   \\
0  & 0  & 0  & 0\%   \\\midrule
D2 & D1 & D0 & Retning \\\midrule
1  & 1  & 1  & \ang{90}      \\
1  & 1  & 0  & \ang{67.5}    \\
1  & 0  & 1  & \ang{45}      \\
1  & 0  & 0  & \ang{22.5}    \\
0  & 1  & 1  & \ang{-90}     \\
0  & 1  & 0  & \ang{-67.5}   \\
0  & 0  & 1  & \ang{-45}     \\
0  & 0  & 0  & \ang{-22.5}   \\\bottomrule
\end{tabular}%
\end{table}

\fig{Strukturering/Styringsenhed/StyringsenhedCirkelUdkast}{.4}{Konceptillustration der viser styringsprotokollen for SumoBot-enhederne}