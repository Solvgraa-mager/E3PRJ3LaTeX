\section{Koncept}
 Kender vi ikke alle problemet med en drukmås der har drukket for mange øller og får lyst til at skrige i en mikrofon? Det hidtil eneste kendte produkt til at afhjælpe dette problem, har været karaoke, dog med den ulempe at alle andre på baren skal høre på det. Robo-Sumo-Battle vil specifikt afhjælpe dette problem ved i stedet at lade personen bruge sin stemme til at styre to små chubby robotter der dyrker sumo-brydning, som samtidig er sjov at se på - vi bruger således en gammel japansk tradition til at afhjælpe problemerne med en nyere japansk tradition. 

Produktet er et spil for to spillere, hvor hver spiller styrer en to-hjulet robot rundt på en bane.

\figOC{Indledning/RigtBillede.png}{0.8}{Rigt billede der viser hvordan produktet er tiltænkt.}


Styringen foregår via én kontrolenhed pr. spiller og vil være baseret på sensorinput i form af lyd igennem en mikrofon og bevægelse igennem et \textit{d-pad}. Dertil kan man aktivere forskellige \textit{gamemodes} hvor der er udviklet mere eller mindre arbitrære styringer for at skrue sværhedsgraden op. 
Disse signaler bliver digitalt behandlet og sendt til robotterne som styrekommandoer via den embedded software i spilplatformen.
Spillet er tænkt udført som en traditionel japansk sumokamp --- de to robotter mødes i en ring og kæmper. Vinderen er den der fratager alle modstanderens liv først - eller den sidste robot i ringen.

\subsection{Spilregler} \label{Spilregler}

\begin{itemize}
    \item De to robotter anbringes på et markeret \textit{startfelt}.
    \item Når robotterne er anbragt korrekt på deres startfelter, kan en nedtælling fra 3 sekunder startes ved, at begge spillere trykker på hver deres knap, hvorefter spillet går i gang.
    \item Hver robot har fra spillets start 3 liv.
    \item Hver runde vare maksimalt 1 minut.
    \item I en rundes varighed gælder:
    \begin{itemize}
        \item Hvis modstanderen skubbes udover den markerede bane, vindes hele spillet.
        \item Ved at påkøre modstanderen bagfra eller fra siden, mister modstanderen et liv.
        \item Hvis en spiller selv bringer sig udover den markerede bane, vinder modstanderen.
    \end{itemize}
    \item Ved tab af liv placeres hver robot ved deres respektive startfelt.
\end{itemize}

Til styringen af robotterne medfølger der til spillet et instrument med veldefinerede frekvensområder: en mundharmonika.
\tbd
%  Til disse instrumenter tilknytter der sig som udgangspunkt følgende kommandoer: frem, tilbage, højre og venstre. Disse kommandoer eksekveres med udgangspunkt i \figref{Indledning/blokfloejte_noder} som følgende: 
% \begin{itemize}
%     \item A: Fremad
%     \item B: Bagud
%     \item C': Venstre 
%     \item D': Højre
% \end{itemize}

% \fig{Indledning/blokfloejte_noder}{0.2}{Illustration der viser tonaliteten af en blokfløjte. Her ses hvordan styretonerne A, B, C og D spilles}

\fig{Strukturering/Konceptillustrationer/Spilleplade_Topview.eps}{1}{Selve spilpladen}
\fig{Strukturering/Konceptillustrationer/Spilleplade_Sideview.eps}{1}{Spillepladen set fra siden}
\fig{Strukturering/Konceptillustrationer/Spilleplade_Frontview.eps}{1}{Spillepladen set fra den ene spillers side}

