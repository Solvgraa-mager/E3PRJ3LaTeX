{
\def\dirfigs{figs/Analyse/Styringsenhed/}
\section{Styringsenhed}\label{sec:Styringsenhed:analyse}
\subsection{Lydgiver}
Der er blevet foretaget en analyse af forskellige lydinputs til systemet. Analysen består af en måleserie, hvor man med forskellige lydinputs skal afgøre, om det er muligt at lave signaldiskrimination i 4 frekvensbånd. Målingerne er foretaget ved følgende procedure:
\begin{itemize}
    \item Indspilning af 4 forskellige toner af 3 omgange.
    \item Lydfilerne eksporteres som .wav fil og indlæses i Matlab.
    \item Lydfilerne processeres i Matlab til at et frekvensspektrum der vurderes muligt til at diskriminere lydinputtet til 4 styringsoutputs er opnået. 
    \item Det endelige resultat afspejler valget af lydinput samt specifikationerne for styringsenheden.
\end{itemize}

\subsection{Mandestemme}
\fig{ToneAnalyse/Voice_FFT_Close_N_512}{.7}{Mandestemme, 4 toners frekvensspektrum}

I Figur \ref{fig:ToneAnalyse/Voice_FFT_Close_N_512} ses et frekvensspektrum af en mandestemme, der for så vidt muligt har forsøgt at generere 4 forskellige toner. Her er påført hann-vindue for at minimere sidesløjfer og foretaget en 256 punkters FFT.

Dette giver anledning til følgende vurdering:
\begin{itemize}
    \item Der sættes alt for høje krav til aktøren om at ramme bestemte toner, hvis der ikke indføres en talegenkendelsesstandard. Alene ud fra denne praktiske overvejelse forkastes stemmer som inputs. 
\end{itemize}

\subsection{Blokfløjte}
\fig{ToneAnalyse/Recorder_FFT_N_128}{.7}{Blokfløjte, 64 bins frekvensspektrum}
\fig{ToneAnalyse/Recorder_FFT_N_128}{.7}{Blokfløjte, 256 bins FFT - ubehandlet}

\subsection{Mundharmonika}
\subsubsection{Rå data}
\fig{ToneAnalyse/NoHannFFT_Close_N_128}{.7}{64 bins FFT - ubehandlet}
\fig{ToneAnalyse/NoHannFFT_Bar_N_128_Bins_64}{.7}{64 bins FFT  - ubehandlet}

\fig{ToneAnalyse/NoHannFFT_Close_N_256}{.7}{128 bins FFT - ubehandlet}
\fig{ToneAnalyse/NoHannFFT_Bar_N_256_Bins_128}{.7}{128 bins FFT - ubehandlet}


\fig{ToneAnalyse/NoHannFFT_Close_N_512}{.7}{256 bins FFT - ubehandlet}
\fig{ToneAnalyse/NoHannFFT_Bar_N_512_Bins_256}{.7}{256 bins FFT - ubehandlet}

\subsubsection{Behandlet data}
\fig{ToneAnalyse/FFT_Close_N_128}{.7}{64 bins FFT - behandlet med Hann-vindue}
\fig{ToneAnalyse/FFT_Bar_N_128_Bins_64}{.7}{64 bins FFT - behandlet med Hann-vindue}

\fig{ToneAnalyse/FFT_Close_N_256}{.7}{128 bins FFT - behandlet med Hann-vindue}
\fig{ToneAnalyse/FFT_Bar_N_256_Bins_128}{.7}{128 bins FFT - behandlet med Hann-vindue}


\fig{ToneAnalyse/FFT_Close_N_512}{.7}{256 bins FFT - behandlet med Hann-vindue}
\fig{ToneAnalyse/FFT_Bar_N_512_Bins_256}{.7}{256 bins FFT - behandlet med Hann-vindue}




% % \tbd En studerende fra DIEM\footnote{Dansk Institut for Elektronisk Musik - Det Jyske musikkonservertiorium} har leveret en række lydoptagelser af hhv. blokfløjte og mundharmonika, der kan bruges til at afklare hvilke frekvensområder der er interessante. % mangler endnu.
% Det antages at frekvensområdet ligger i området \SIrange{300}{3000}{Hz}\cite{MimsEngNotebookFormulas}. At designe et velfungerende filter i dette frekvensområde burde være muligt uden de store armbevægelser.
% %%% 

\subsection{Mikrofon}
\fig{styringsenhed_mikrofon}{0.4}{Udsnit af mikrofonblokken}
\begin{PartBlokDescription}{Styringsenhed: Mikrofon}{Styringsenhed:Mikrofon}
\Blokbeskrivelse{Mikrofon}{Konvertering fra akustisk til elektrisk signal}
\Portbeskrivelse{Mikrofon}{Analog}{}{
\item Udgangsimpedans: 2.2\kohm 
\item Sensitivitet: \SI{8(2)}{\milli\volt\per\pascal}
\item SNR: 40dBV
}
\Portbeskrivelse{SpillersLyd}{Akustisk}{}{
\item Dynamisk område: 70dB SPL
\item Signalfrekvensområde: 250 til 1.55 kHz
}
\end{PartBlokDescription}

Der er blevet foretaget et valg om at benytte en elektretmikrofon \tbr (link til mikrofon) som allerede var til rådighed. 
Mest kritisk for mikrofonen er SNR\cite{Lewis2012}. Hvis denne ikke viser sig tilstrækkelig høj til at kunne differentiere baggrundsstøj fra det reelle input vil der skulle laves akustisk regulering.

Outputtet fra mikrofonen behandles og konditioneres, hvorfor kravene til Analog signalbehandling og Signalkonditionering blokkene defineres ud fra mikrofonens output.

\subsection{Analog signalbehandling}
\fig{styringsenhed_analogsignalbehandling}{0.4}{Udsnit af blokken til analog signalbehandling}
\begin{PartBlokDescription}{Styringsenhed: Analog signalbehandling}{Styringsenhed:Analogsignalbehandling}
\Blokbeskrivelse{Analog signalbehandling}{Forstærker mikrofon inputtet i en grad at ADC kan konvertere de forskellige værdier. \newline Gain: 250gg}
\Portbeskrivelse{Lydsignal}{Analog}{Ufiltreret og forstærket signal indeholdende lydinformation.}{
\item Indgangsimpedans: Høj
}
\Portbeskrivelse{Behandlet lydsignal}{Analog}{Filtreret og forstærket signal indeholdende lydinformation.}{
\item Signalfrekvensområde: \SIrange{300}{3000}{\Hz}
\item Udgangsimpedans: Lav
}
\end{PartBlokDescription}


Denne blok skal filtrere frekvenser udenfor pasbåndet fra samt forstærke signalet fra mikrofonen op, så det kan registreres af en ADC.

Det er nødvendigt at have et filter der uden problemer kan lukke frekvenser i området \SIrange{300}{3000}{Hz} igennem, uden at det opstår bemærkelsesværdi forvrængning eller faseproblemer, derudover bør uønskede frekvenser bortfiltreres i en grad af de ikke opfanges af ADC'en.

Med et pasbånd beliggende i området \SIrange{300}{3000}{Hz} og en dertilhørende forstærkning på \ca{300}, vil et passende frekevensrespons være et båndpasfilter der kan beskrives : 
\begin{equation}
    T(s) = T_{\textrm{HP}} \cdot T_{\textrm{LP}} \cdot T_{\textrm{Gain}} 
\end{equation}
hvor følgende gør sig gældende for de forskellige led
\begin{equation}
    \begin{split}
        T_{\textrm{HP1}} &= \fdb = \SI{300}{Hz} \quad \SI{20}{\dB\per dec} \\
        T_{\textrm{LP1}} &= \fdb = \SI{3000}{Hz} \quad  \SI{40}{\dB\per dec} \\
        T_{\textrm{Gain}} &= \textrm{Gain} \approx  300
    \end{split}
\end{equation}

\fig{StyringsFlow}{0.3}{Flow der viser funktionaliteten af filteret}

Der ses bort fra \SI{-3}{\dB} dæmpningen ved de to knækfrekvenser, da denne på nuværende tidspunkt ikke vurderes at have en større betydning --- dette korrigeres hvis det volder problemer ifb. ADC-konverteringen.

Ved at lade lavpasledet dæmpe med \SI{40}{\dB\per dec} ved 3\kHz, bør aliaseringsproblemer kunne undgåes når signal sendes videre til ADC'en.

For at kunne aflæse signalet korrekt på ADC'en er signalet nødt til at fluktuere omking et fast punkt, hvor der her vælges \(\frac{V_{CC}}{2}\), hvorved et med der svinger m. 0 til VCC. 
Ændring af amplitude samt evt. ekstra \emph{offsetbiasing} bør finde sted i signalkonditioneringen.

For at fastsætte en endelig gain er en reel test af mikrofonen nødvendig, dog antages det at 
\begin{equation}
    \textrm{Mikrofon spændingsoutput}\approx \SI{8}{mV} \tbr
\end{equation}

Her af
\begin{equation}
    \frac{\textrm{Maksimal spænding}}{\textrm{Mikrofon output}} = \textrm{Gain}
\end{equation} 

Kan niveauet fastsættes udfra en antagelse om en outputspænding fra mikrofonen på \SI{8(2)}{mV} 
\begin{equation}
    \begin{split}
        \textrm{Gain} &\approx  300\\
       300 \cdot \SI{8}{mV} &= \SI{2.4}{V}
    \end{split}
\end{equation}
'Ved nærmere analyse af den valgte mikrofon egenskaber, kan gainniveauet justeres efter behov --- dette bør kunne udføres uden nogle vanskeligheder i et senere forløb.

Outputimpedansen for signalbehandlingsmodulet har ikke den store betydning, men bør holdes i omegnen af \SIrange{1}{10}{\kohm} for ikke at skabe probelmer med videre interfacing af moduler. 




\subsection{Signalkonditionering}
\fig{styringsenhed_signalkonditionering}{0.4}{Udsnit af blokken til signalkonditionering}
DET HER ER FUCKING SIGNALKONDITIONERING

\begin{PartBlokDescription}{Styringsenhed: Signal konditionering}{Styringsenhed:Signalkonditionering}
\Blokbeskrivelse{Signal\newline kondition-ering}{Buffer stage der skal sikre maksimal spændings-overførsel.}
\Portbeskrivelse{Behandlet lydsignal}{Analog}{}{
\item Indgangsimpedans: 10x udgangsimpedans for Analog Signalbehandling blokken.
}
\end{PartBlokDescription}

\subsection{ADC}
\fig{styringsenhed_adc}{0.4}{Udsnit af ADC-blokken}

\begin{PartBlokDescription}{Styringsenhed: ADC}{Styringsenhed:ADC}
\Blokbeskrivelse{ADC}{Analog til digital konvertering.}
\Portbeskrivelse{ADCinLyd}{Analog}{Delta Sigma ADC}{
\item Fuld skala input spændingsområde: \SIrange{0}{5}{\volt}
\item Indgangsimpedans: høj
\item Sample rate: 10 kHz.
\item Bitopløsning: 8 bit
\item SNR: 40 dB
}
\end{PartBlokDescription}


Pr. styringsenhed er der én ADC-blok som konverterer inputs fra hhv. joystickX, joystickY og mikrofon input til bitstrømme. Disse bitstrømme processeres i en microprocessor hvor der kører en realtids-FFT algoritme for at kunne generere en veldefineret styringsprotokol for SumoBotsne. 

For at gøre dette muligt er der udarbejdet følgende krav til ADC modulet:
\begin{itemize}
    \item Samplingrate: 10 kHz
    \item Bitopløsning: 8 bit
    \item Fuld-skala input range: 5V
    \item SNR: 40 dB
    \item Høj indgangsimpedans
\end{itemize}

Fuld skala input range på ADC'ens indgang skal stemme overens med inputsignalets spændingsområde.
Derfor fastsættes at fuld-skala input range = 5V.

\tbr Hvis SNR på 40 dB vurderes tilstrækkeligt, kan denne værdi bruges som udgangspunkt til et mindstekrav for bitopløsningen,\cite{KesterMT-001Care}, som ADC'en skal outputte med:
\[ 40 = 6.02 \cdot n+1.76 \]
\[ n = 6.37 \to n = 8 \]

Der skal altså anvendes en ADC med en effektiv bitopløsning på mindst 6.37. Der vælges derfor at anvende en ADC med 8 bit opløsning. 

Til lydinputtet fra mikrofonforstærkeren bestemmes det, at der skal anvendes en Delta Sigma ADC. Dette er bestemt med følgende grundlag:
\begin{itemize}
    \item Forhåndsforsikring for, at ADC'en kan yde den tilstrækkelige bitopløsning som inputs til en FFT algoritme. 
    \item Delta Sigma ADC opererer kontinuerligt(\tbr reference til Texas Instruments) (integrerer over tid) og er derfor bedst til realtids-applikationer.  
\end{itemize}

For at opfylde kravene til blokken ADC er der gjort overvejelser, som sammen med en vurdering ud fra kravene er opstillet i tabel \ref{tab:adc_vurdering}. Da PSoC 5LP opfylder alle krav, vælges denne. 

\subsection{Digital signal processing}
\fig{styringsenhed_dsp}{0.4}{Udsnit af ADC-blokken}
For ikke selv at skulle designe ADC'er analogt tages udgangspunkt i en platform hvor der kan køres noget firmware med en FFT-algoritme til lydsignalet fra mikrofonen. Her er til PSoC 5LP udarbejdet et bibliotek til FFT\tbr, som der i første omgang tages udgangspunkt i. Denne kan lave en 128 punkters FFT hvilket vurderes tilstrækkeligt i denne applikation med en konservativ signalbåndbredde. Derved kan signalprocesseringen også foregå i selvsamme blok. 
}






















