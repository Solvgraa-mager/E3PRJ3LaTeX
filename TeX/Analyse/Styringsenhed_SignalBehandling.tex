Denne blok skal filtrere frekvenser udenfor pasbåndet fra samt forstærke signalet fra mikrofonen op, så det kan registreres af en ADC.

Det er nødvendigt at have et filter der uden problemer kan lukke frekvenser i området \SIrange{300}{3000}{Hz} igennem, uden at det opstår bemærkelsesværdi forvrængning eller faseproblemer, derudover bør uønskede frekvenser bortfiltreres i en grad af de ikke opfanges af ADC'en.

Med et pasbånd beliggende i området \SIrange{300}{3000}{Hz} og en dertilhørende forstærkning på \ca{300}, vil et passende frekevensrespons være et båndpasfilter der kan beskrives : 
\begin{equation}
    T(s) = T_{\textrm{HP}} \cdot T_{\textrm{LP}} \cdot T_{\textrm{Gain}} 
\end{equation}
hvor følgende gør sig gældende for de forskellige led
\begin{equation}
    \begin{split}
        T_{\textrm{HP1}} &= \fdb = \SI{300}{Hz} \quad \SI{20}{\dB\per dec} \\
        T_{\textrm{LP1}} &= \fdb = \SI{3000}{Hz} \quad  \SI{40}{\dB\per dec} \\
        T_{\textrm{Gain}} &= \textrm{Gain} \approx  300
    \end{split}
\end{equation}

\fig{StyringsFlow}{0.3}{Flow der viser funktionaliteten af filteret}

Der ses bort fra \SI{-3}{\dB} dæmpningen ved de to knækfrekvenser, da denne på nuværende tidspunkt ikke vurderes at have en større betydning --- dette korrigeres hvis det volder problemer ifb. ADC-konverteringen.

Ved at lade lavpasledet dæmpe med \SI{40}{\dB\per dec} ved 3\kHz, bør aliaseringsproblemer kunne undgåes når signal sendes videre til ADC'en.

For at kunne aflæse signalet korrekt på ADC'en er signalet nødt til at fluktuere omking et fast punkt, hvor der her vælges \(\frac{V_{CC}}{2}\), hvorved et med der svinger m. 0 til VCC. 
Ændring af amplitude samt evt. ekstra \emph{offsetbiasing} bør finde sted i signalkonditioneringen.

For at fastsætte en endelig gain er en reel test af mikrofonen nødvendig, dog antages det at 
\begin{equation}
    \textrm{Mikrofon spændingsoutput}\approx \SI{8}{mV} \tbr
\end{equation}

Her af
\begin{equation}
    \frac{\textrm{Maksimal spænding}}{\textrm{Mikrofon output}} = \textrm{Gain}
\end{equation} 

Kan niveauet fastsættes udfra en antagelse om en outputspænding fra mikrofonen på \SI{8(2)}{mV} 
\begin{equation}
    \begin{split}
        \textrm{Gain} &\approx  300\\
       300 \cdot \SI{8}{mV} &= \SI{2.4}{V}
    \end{split}
\end{equation}
'Ved nærmere analyse af den valgte mikrofon egenskaber, kan gainniveauet justeres efter behov --- dette bør kunne udføres uden nogle vanskeligheder i et senere forløb.

Outputimpedansen for signalbehandlingsmodulet har ikke den store betydning, men bør holdes i omegnen af \SIrange{1}{10}{\kohm} for ikke at skabe probelmer med videre interfacing af moduler. 